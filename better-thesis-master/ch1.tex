\chapter{Data}
\label{chap:data}
Here we go, the first chapter! We need to be checked by grammarly
Data v naší aplikaci hrají klíčovou roli, a proto se v následující kapitole budeme podrobně zabývat jejich dostupností. Následně provedeme diskuzi o možnostech řešení této problematiky.

Při volbě zdroje dat je dobré zvážit několik aspektů, které jsou pro nás důležité. Mezi tyto aspekty patří:
\begin{description}
    \item[Spolehlivost] Vyjadřuje míru, do jaké je možné důvěřovat danému zdroji na základě jeho historie a reputace.
\end{description}
Existují API třetích stran, avšak těm se budeme chtít vyhnout z několik důvodů... sbírají data z několika zdrojů a může se stát, že o jednom tématu zachytíme více zpráv, avšak budou pojednávat o stejném tématu. Tím pádem bychom měli v naší databázi redundantní data, která by mohla způsobit problémy při vyhledávání. Dalším důvodem je, že některá API jsou placená a my bychom chtěli mít aplikaci zdarma.
\section{Archiv/aktuální}
\section{Dostupnost}


