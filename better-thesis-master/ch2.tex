\chapter{Related work}
\label{chap:related-work}
While examining online applications with similar characteristics, a common challenge arises from the need for more transparency regarding the techniques applied in sentiment analysis. Specifically, employed methods and the details of the models remain undisclosed. We acknowledge this limitation due to the proprietary nature of the software, mainly as it is an essential part of the business model. Hence, this chapter is organized into two primary sections. 

The initial part provides an overview of existing applications. At the same time, the second section will concern the most recent and relevant research on predicting stock market behaviour using data mining techniques and news sentiment analysis. Additionally, it explores research studies on entity-level sentiment analysis over news articles and its application in the stock market.

\section{Existing Application Overview}
\label{sec:existing-application-overview}

Sentiment analysis applications accessible to the public are typically based on investigating social network posts, with StockTwist\footnote{\href{https://www.stocktwits.com}{\textit{https://www.stocktwits.com}}} as a notable illustration. According to \textcite{reuters2022}\todo{Q-2.1 Lze takto citovat "pouze" webový článek deníku? Url je moc velký a chtěl bych se vyhnout tomu, abych ho tlačil do footnote}, StockTwits in 2022 boasts more than six million registered users and one million monthly active users, underscoring its prominent user base. With the growing volume of social media contributions, it is difficult to determine which post will prompt action. In our case, these posts do not constitute highly relevant data since our interest is in news articles containing a more significant amount of information. Even if the selection of accounts is limited to informative sources, News organizations' social media accounts only link to their articles, usually accompanied by a headline or lead paragraph. These posts are not enough for our purposes, as we need the entire article to perform better context for entity-level sentiment analysis (discussed further in Chapter \ref{chap:textual-data}).

\subsection{Bloomberg Terminal}
\label{subsec:bloomberg-terminal}

The only software similar to the one created is a module in the so-called Bloomberg Terminal\footnote{\href{https://www.bloomberg.com/professional/solution/bloomberg-terminal/}{\textit{https://www.bloomberg.com/professional/solution/bloomberg-terminal/}}} from Bloomberg L.P.. This software system provides investors with analytical tools over financial data, including sentiment analysis of news articles and posts on social network X, formerly known as Twitter. The cost of a\todo{TODO: a nebo the ?} Bloomberg Terminal depends on the required specific features and services. \linebreak A standard subscription typically amounts to approximately \$24,000 per year. It is a very complex and powerful software, but it is not accessible to the general public.

Bloomberg, in its work \parencite{bloomberg}, describes two types of sentiment analysis: story-level and company-level sentiment, utilizing a suite of \acrfull{sml} techniques. Classification engines are trained on labeled datasets containing news articles and social media posts. Reportedly, the labeling process is based on the question:\footnote{This information is difficult to verify due to the unavailability of the dataset.} \begin{quote}
    \textit{``If an investor having a long position in the security mentioned were to read this news or tweet, is he/she bullish, bearish or neutral on his/her holdings?''}
\end{quote} Once model training is completed, the models are employed to analyze recently published posts and articles associated with organizations, seeking distinctive sentiment signals related to the business and finance domain. As mentioned above, sentiment is divided into two levels:\begin{description}
    \item[Story-level] Sentiment score value of articles and posts is calculated after arrival in real time. The calculation includes score and confidence, where the score has one of three options: positive, negative, or neutral, each described by a numerical value from the set $\{-1, 0, 1\}$. The confidence is defined by a~value ranging from $0$ to $1$, demonstrating the intensity\footnote{Probability of being positive, negative, or neutral} of the sentiment. Hence, the story-level sentiment ranges from $-1$ to $1$. For both, we get the following equation:
    \begin{equation}
    \label{eq:story-level-article-sentiment}
    \text{\textit{Story-level}}_{c}^{\text{\textit{Articles}}} = S_{c}^{a} C_{c}^{a}, \quad a \in P(c)
    \end{equation}
    \begin{equation}
    \label{eq:story-level-post-sentiment}
    \text{\textit{Story-level}}_{c}^{\text{\textit{Posts}}} = S_{c}^{p} C_{c}^{p}, \quad p \in P(c)
    \end{equation} where $a$ represents an article and $p$ represents a post from $P(c)$, the set of published articles and posts referencing company $c$. $S_{c}^{a}$ and $S_{c}^{p}$ are the sentiment polarity scores of article $a$ and post $p$ that reference company $c$. $C_{c}^{a}$ and $C_{c}^{p}$ are the confidences of article $a$ and post $p$ that reference company $c$.
    \item[Company-level] Sentiment score value is then calculated as the confidence-weighted average of the story-level sentiment scores, incorporating all relevant news articles and social media posts mentioning the company.
    \begin{equation}
        \label{eq:company-level-article-sentiment}
        \text{\textit{Company-level}}_{c,t}^{Articles} = \frac{\sum_{a \in P(c,T)} S_{c}^{a} C_{c}^{a}}{N_{c,T}^{a}}, \quad T \in [t_{b}, t]
    \end{equation}
    \begin{equation}
        \label{eq:company-level-post-sentiment}
        \text{\textit{Company-level}}_{c,t}^{Posts} = \frac{\sum_{p \in P(c,T)} S_{c}^{p} C_{c}^{p}}{N_{c,T}^{p}}, \quad T \in [t_{b}, t]
    \end{equation} where $a$ represents an article and $p$ represents a post from $P(c,T)$, the set of published articles and posts referencing company $c$ during period $T$. Period $T$ is a time interval of length $t_{b}$ to $t$, where $t_{b}$ is the time constant of the beginning. $N_{c,T}^{a}$ and $N_{c,T}^{p}$ are the number of articles and posts referencing company $c$ during period $T$. In this way, the company-level sentiment is calculated as follows:
    \begin{equation}
    \label{eq:company-level-sentiment}
    \text{\textit{Company-level}}_{c,t} = \text{\textit{Company-level}}_{c,t}^{Articles} + \text{\textit{Company-level}}_{c,t}^{Posts}
    \end{equation}
 \end{description}

Intraday Company-level sentiment score for news articles is recalculated every two minutes, utilizing an eight-hour rolling window. The sentiment score for social network posts is recalculated every minute, employing a 30-minute rolling window. Due to the previous definitions, we can express these by a simple substitution of $t_{b}$ depending on the rolling window as follows:
\begin{equation}
    \label{eq:intraday-company-level-article-sentiment}
    \text{\textit{Intraday Company-level}}_{c,t}^{Articles} = \frac{\sum_{a \in P(c,T)} S_{c}^{a} C_{c}^{a}}{N_{c,T}^{a}}, \quad T \in [t-8, t]
\end{equation}
\begin{equation}
    \label{eq:intraday-company-level-post-sentiment}
    \text{\textit{Intraday Company-level}}_{c,t}^{Posts} = \frac{\sum_{p \in P(c,T)} S_{c}^{p} C_{c}^{p}}{N_{c,T}^{p}}, \quad T \in [t-0.5, t]
\end{equation}
\begin{equation}
    \label{eq:intraday-company-level-sentiment}
    \text{\textit{Intraday Company-level}}_{c,t} = \text{\textit{Company-level}}_{c,t}^{Articles} + \text{\textit{Company-level}}_{c,t}^{Posts}
\end{equation}

Daily company-level sentiment scores are published every morning about 10 minutes before the market opens. The calculation is determined as a confidence-weighted average of sentiment scores derived from the story-level sentiments of news and social media posts over the past 24 hours as follows:
\begin{equation}
    \label{eq:daily-company-level-sentiment}
    \text{\textit{Daily Company-level}}_{c,t} = \frac{\sum_{d \in P(c,T)} S_{c}^{d} C_{c}^{d}}{N_{c,T}}, \quad T \in [t-24, t]
\end{equation} where document $d$ represents a news article or social media post, the sentiment polarity score of document $d$ referencing company $c$ is denoted as $S_{c}^{d}$, and the confidence associated with this reference is represented by $C_{c}^{d}$. The set $P(c,T)$ encompasses non-neutral documents referencing company $c$ published within the last 24 hours. $N_{c,T}$ expresses the count of non-neutral documents referencing company $c$ during period $T$. This approach is further explored in terms of the informational role of social media by \textcite{chenInformationalRole}.

\section{Predicting Stock Market Behaviour}
\label{sec:predicting-stock-market-behaviour}
%https://www.sciencedirect.com/science/article/pii/S0378437119320904 poukaz na hypotézy a pohledy na danou problematiku
Several approaches for predicting stock market behaviour and price trends have been proposed, utilizing sentiment analysis of financial news and historical stock prices. Several studies prove a strong correlation between financial news sentiment and stock prices \parencite{li2014newsimpact} \parencite{Wan2021}. Due to the nature of unstructured textual data, predicting stock market behaviour is a challenging task.

\textcite{khedr2017predicting} focuses on creating an effective model for forecasting future trends in the stock market, using sentiment analysis of financial news and historical stock prices. The model achieves more accurate results than previous works by considering different market and company news types combined with historical stock prices. The experiments utilize datasets from three companies: Yahoo Inc., Facebook Inc.\footnote{Known as Meta since 2021}, and Microsoft Corporation. The authors use well-known and informative news sources such as Reuters, The Wall Street Journal, and Nasdaq. The first step of sentiment analysis to get the text polarity using a Naive Bayes classifier was shown to achieve accuracy from 72.73\% to 86.21\%, while the second step, which combines news sentiment with historical prices, improved prediction accuracy up to 89.80\%. Moreover, that is why we find this study motivating and inspiring.

\section{Entity-level Sentiment Analysis}
\label{sec:entity-level-sentiment-analysis}
This section will discuss the most recent and relevant research conducted in entity-level sentiment analysis and its application in news articles, including named entity recognition over news articles.

\cite{zhao2021bert} employed RoBERTa,  a Robustly Optimized \acrfull{bert} Pretraining Approach \parencite{liu2019roberta}, to propose a sentiment analysis and entity detection strategy in financial text mining and public opinion analysis in social media. In the first step, sentiment analysis, mainly focusing on negative polarity, is performed. Then, entity detection is considered in different granularities with MRC, Machine Reading Comprehension \parencite{liu2019mrc}, or sentence-matching tasks. As a result, this study serves entity detection differently than traditional Named Entity Recognition. The authors claim that the proposed method outperforms traditional sentiment analysis and entity detection methods. The authors also emphasize the importance of the financial domain, where the sentiment of a single entity can significantly affect the stock price.

Named entity recognition plays a significant role in entity-level
sentiment analysis, as discussed in Section \ref{sec:entity-level-sentiment-analysis}. \textcolor{lightgray}{Vybírám články, které stojí za zmínku.}

\section{Mining dynamic Social Networks}
\label{sec:mining-dynamic-social-networks}
\todo{Ještě si nejsem jistý, zda-li je tato sekce potřebná.}
This section will discuss interesting research \cite{Jin2012} dealing with mining dynamic social networks and its application in the stock market. 

%https://lbezone.hkust.edu.hk/bib/991012657169103412